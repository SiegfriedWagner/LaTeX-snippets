\documentclass[a4paper, 12pt]{report}
\usepackage{polski}
\usepackage[T1]{fontenc} % rozszerzony zestaw znaków
\usepackage[utf8]{inputenc} % wsparcie kodowanie unicode
\usepackage[polish]{babel}
\usepackage[left=3.5cm,top=2.5cm,right=2.5cm,bottom=2.5cm]{geometry} % konfiguracja marginesów
% \usepackage{layout}
\usepackage{natbib}
\usepackage{hyperref}
% \usepackage[numbers]{natbib} % powraca do standardowej metody cytowania
\begin{document}
% biblatex do cytowań w stopce \footcite{}
\noindent To jest minimalny plik z cytatem \\
pierwszym \cite{de1999philosophy} \\
 cytowanie citep \citep{johnson1980mental} \\
trzecim \cite{simon1980cognitive} \\
i czwartym (citet) \citet{norman1980twelve}. \\
W bibliografii pojawiają sie tylko te wpisy z bliku biblio, ktore zostaly uzyte w pracy.
A tak się uzywa stopki\footnote{Przykładowa stopka}.
\begin{tabbing}
  $\backslash$citep[chap.~2]\{de1999philosophy\}dddd \qquad\= $\Rightarrow$ \qquad\= \citep[chap.~2]{de1999philosophy} \kill
  \verb!\citet{nazwa}! \> $\Longrightarrow$ \> \citet{de1999philosophy} \\
  \verb!\citet[chap.~2]{de1999philosophy}! \> $\Longrightarrow$ \> \citet[rodzdział.~2]{de1999philosophy} \\
  \verb!\citep{nazwa}! \> $\Longrightarrow$ \>  \citep{de1999philosophy} \\
  \verb!\citep[chap.~2]{de1999philosophy}! \> $\Longrightarrow$ \> \citep[chap.~2]{de1999philosophy} \\
  \verb!\citep[see][]{nazwa}! \> $\Longrightarrow$ \> \citep[zobacz][]{de1999philosophy} \\
  \verb!\citet*{nazwa}! \> $\Longrightarrow$ \> \citet*{de1999philosophy} \\
  \verb!\citep*{nazwa}! \> $\Longrightarrow$ \> \citep*{de1999philosophy} \\
\end{tabbing}
% more styles at https://www.overleaf.com/learn/latex/Bibtex%20bibliography%20styles#Biblatex_styles
\bibliographystyle{apalike}
\bibliography{biblio}

\end{document}
